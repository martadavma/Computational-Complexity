\documentclass{cis320}

\HWauthor{Marta Davila Mateu}{davilama@usc.edu}
\HWauthor{Yifei Huang}{yifeih@usc.edu}
\HWno{1}
\HWcourse{CSCI 599}
%\HWextension

\begin{document}
\maketitle




\HWproblem
%\HWsubproblem
\textit{Suppose} \( L_1, L_2 \in NP \). \textit{Please prove (or disprove) that }\( L_1 \cup L_2 \in NP \) \textit{and} \(L_1 \cap L_2 \in NP\).

\:

Assume \( L_1, L_2 \in NP \).

\textbf{First we will prove that \( L_1 \cup L_2 \in NP \).}

Assume \( x \) is the input string we want to verify for \( L_1 \cup L_2\).

The \( L_1 \cup L_2\) verifier would choose whether to verify \(L_1\) or \(L_2\) non-deterministically.

For the chosen language (\(L_1\) or \(L_2\)), the verifier would use the non-deterministic choices to guess a certificate (a string that helps prove that \( x \) is in the language) and then verify it in polynomial time.

If x is in either \(L_1\) or \(L_2\), there must exist a certificate that proves this, and the verifier can non-deterministically guess and verify it in polynomial time.

Since \(L_1\) and \(L_2\) are both in NP, there exist non-deterministic polynomial-time verifiers for both. Therefore, you can combine these verifiers to create a non-deterministic polynomial-time verifier for \( L_1 \cup L_2\), therefore \( L_1 \cup L_2\) must be in NP.

\textbf{Now we will proceed to prove that \(L_1 \cap L_2 \in NP\).}

Assume \( y \) is the input string we want to verify for \( L_1 \cap L_2\).

The \( L_1 \cap L_2\) verifier would choose whether to verify \(L_1\) or \(L_2\) non-deterministically

For the chosen language (either \(L_1\) or \(L_2\)), the verifier would use the non-deterministic choices to guess a certificate (a string that helps prove that y is in the language) and then verify it in polynomial time.

If y is in both \(L_1\) and \(L_2\), there must exist certificates that prove this for both languages, and the verifier can non-deterministically guess and verify both certificates in polynomial time.

Since \(L_1\) and \(L_2\) are both in NP, there exist non-deterministic polynomial-time verifiers for both. 

Therefore, we can combine these verifiers to create a non-deterministic polynomial-time verifier for \( L_1 \cap L_2\), proving that \( L_1 \cap L_2\) is in NP.
\bigskip




\HWproblem
\textit{Assume that SAT := \{$\phi$  : $\phi$ is a satisfiable CNF\} is NP-complete. Prove that 3SAT := \{$\phi$ : $\phi$  is a satisfiable 3-CNF\} is NP-complete.}

\:

To prove that if SAT is NP-complete then 3SAT is NP-complete, we will reduce SAT to 3SAT.

Given an instance of SAT with variables $x_1, x_2, ..., x_n$ and clauses $C_1, C_2, ..., C_m$, you can construct an equivalent instance of 3SAT as follows:

For each clause $C_i$ in SAT, if it has three literals, we keep it unchanged in 3SAT. If $C_i$ has more than three literals, we create new auxiliary variables $y_1, y_2, ..., y_k$ to break down the clause into clauses with exactly three literals. For example, if we have a clause like ($x_1$ OR $x_2$ OR $x_3$ OR $x_4$), we can break it into two 3SAT clauses: ($x_1$ OR $x_2$ OR $y_1$) and ($y_1$ OR $x_3$ OR $x_4$).

Then we repeat step 1 for all clauses in the SAT instance.

If we introduced any new auxiliary variables $y_1, y_2, ..., y_k$ in step 1, we have to add clauses to ensure that each $y$ variable is assigned exactly one value.

This reduction from SAT to 3SAT preserves the satisfiability of the problem because if the original SAT instance is satisfiable, then the equivalent 3SAT instance will also be satisfiable. Conversely, if the 3SAT instance is satisfiable, it means that all the clauses, whether they were originally 3 literals or transformed to 3 literals, are satisfiable, so the original SAT instance is also satisfiable. Therefore, the resulting instance of 3SAT is equivalent to the original SAT instance. 

This reduction demonstrates that if SAT is NP-complete then 3SAT is NP-complete, as SAT can be reduced to 3SAT in polynomial time. Therefore, solving 3SAT is as hard as solving SAT.
\bigskip




\HWproblem
Prove that \( SPACE(n^3) \not\subseteq SPACE(n^2) \)

\end{document}
